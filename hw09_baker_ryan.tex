\documentclass{article}

\usepackage{listings}

\title{ECE 40400 Homework 9}
\date{\today}
\author{Ryan Baker}

\begin{document}
\maketitle

\begin{enumerate}

\item Requirement 1

\noindent
Requirement 1 mandates that we flush rules and delete custom chains in the filter table:

\begin{lstlisting}[basicstyle=\ttfamily,frame=single]
iptables -F
iptables -X
\end{lstlisting}

\noindent
\texttt{-F} flushed all existing rules in all chains. \texttt{-X} deletes any chains.

\item Requirement 2

\noindent
Reject packets from f1.com:

\begin{lstlisting}[basicstyle=\ttfamily,frame=single]
iptables -A input -s f1.com -j REJECT
\end{lstlisting}

\noindent
\texttt{-A INPUT} appends an input command to the chain. \texttt{-s f1.com} specifies the source of input packets. \texttt{-j REJECT} rejects packets from the source.

\item Requirement 3

\noindent

Enable MASQUERADE for outgoing packets:

\begin{lstlisting}[basicstyle=\ttfamily,frame=single]
iptables -t nat -A POSTROUTING -o eth0
	-j MASQUERADE
\end{lstlisting}

\noindent
\texttt{-t nat} specifies the NAT table.
\texttt{-A POSTROUTING} appends the rule to the post-routing chain.
\texttt{-o eth0} applies NAT only to packets leaving via eth0.
\texttt{-j MASQUERADE} replaces source IP with the machine's public IP.

\item Requirement 4

\noindent
Protect against port scanning:

\begin{lstlisting}[basicstyle=\ttfamily,frame=single]
iptables -N PORTSCAN
iptables -A PORTSCAN -p tcp --tcp-flags
	SYN,ACK,FIN,RST RST -m limit --limit 1/s
	-j RETURN
iptables -A PORTSCAN -j DROP
iptables -A INPUT -p tcp --tcp-flags ALL SYN,ACK
	-j PORTSCAN

\end{lstlisting}

\noindent
\texttt{-N PORTSCAN} makes a new chain called PORTSCAN.
\texttt{-p tcp --tcp-flags SYN,ACK,FIN,RST RST} detects various TCP packets.
\texttt{-m limit --limit 1/s} limits accepted packets to 1 per second.
\texttt{-j RETURN} lets proper traffic pass if it is below the limit.
\texttt{-j DROP} discards excessive scans.
\texttt{-A INPUT -p tcp --tcp-flags ALL SYN,ACK -j PORTSCAN} directs suspected scan traffic to the PORTSCAN chain.

\item Requirement 5

\noindent
Protect against SYN-Flooding:

\begin{lstlisting}[basicstyle=\ttfamily,frame=single]
iptables -A INPUT -p tcp --syn 
	-m limit --limit 1/s --limit-burst 500
	-j ACCEPT
iptables -A INPUT -p tcp --syn -j DROP
\end{lstlisting}

\noindent
\texttt{-p tcp --syn} filters for SYN packets.
\texttt{-m limit --limit 1/s --limit-} \texttt{burst 500} allows only 1 new connection per second after the first 500 connections.
This allows legitimate connections while dropping excessive ones.

\item Requirement 6

\noindent
Allow full loopback access:

\begin{lstlisting}[basicstyle=\ttfamily,frame=single]
iptables -A INPUT -i lo -j ACCEPT
iptables -A OUTPUT -o lo -j ACCEPT
\end{lstlisting}

\noindent
\texttt{-i lo} and \texttt{-o lo} ensure loopback access to localhost.

\item Requirement 7

\noindent
Forward traffic from port 8888 to 25565:

\begin{lstlisting}[basicstyle=\ttfamily,frame=single]
iptables -t nat -A PREROUTING -p tcp --dport
	8888 -j DNAT --to-destination :25565
\end{lstlisting}

\noindent
\texttt{-t nat} specifies the NAT table.
\texttt{-p tcp --dport 8888} modifies packets coming for 8888.
\texttt{-j DNAT --to-destination :25565} redirects them to port 25565.

\item Allow SSH connections only to engineering.purdue.edu:

\begin{lstlisting}[basicstyle=\ttfamily,frame=single]
iptables -A OUTPUT -p tcp --dport 22 -d 
	engineering.purdue.edu -m state
	--state NEW,ESTABLISHED -j ACCEPT
iptables -A INPUT -p tcp --sport 22 -s 
	engineering.purdue.edu -m state
	--state ESTABLISHED -j ACCEPT
\end{lstlisting}

\noindent
Port 22 is SSH. \texttt{-d engineering.purdue.edu} restricts access to only engineering.purdue.edu. \texttt{--state NEW,ESTABLISHED} allows new and ongoing connections.

\item Requirement 9

\noindent
Drop all other packets:

\begin{lstlisting}[basicstyle=\ttfamily,frame=single]

\end{lstlisting}
iptables -A INPUT -j DROP
iptables -A FORWARD -j DROP
iptables -A OUTPUT -j DROP
\end{enumerate}

\noindent
Disallows traffic from being accepted, forwarded, or sent.

\end{document}